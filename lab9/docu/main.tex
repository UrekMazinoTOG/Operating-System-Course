\documentclass[12pt]{article}
\usepackage[english]{babel}
\usepackage[utf8]{vietnam}
\usepackage[margin=2cm]{geometry}
\usepackage[unicode]{hyperref}
\usepackage{multicol}

\usepackage{fancyhdr}
\pagestyle{fancy}
\fancyhf{}
\rhead{Operating Systems Course}
\lhead{Lab 8}
\rfoot{Page \thepage}
\lfoot{\leftmark}

\begin{document}

\begin{flushleft}
    \begin{tabular}{c}
        \\ \\ \hline \\
        \multicolumn{1}{l}{\textbf{{\Huge Lab 9 Paging}}}
        \\ \\
        \textbf{{\Huge Course: Operating Systems}}
        \\ \\ \hline \\
    \end{tabular}
\end{flushleft}

\selectlanguage{english}

\vspace{0.5cm}

\section*{Exercise 1}
Consider the page table shown in Figure 3.1 for a system with 12-bit virtual andphysical addresses and with 256-byte pages. The list of free page frames is D, E, F (that is, D is at the head of the list, E is second, and F is last). Convert the following virtual addresses to their equivalent physical addresses in hexadecimal. All numbers are given in hexadecimal. (A dash for a page frame indicates that the page is not in memory.)


\begin{multicols}{3}


    \begin{tabular}{|c|c|}
    \hline
    Page & Page frame \\ \hline
    0    & \_         \\ \hline
    1    & 2          \\ \hline
    2    & C          \\ \hline
    3    & A          \\ \hline
    4    & \_         \\ \hline
    5    & 4          \\ \hline
    6    & 3          \\ \hline
    7    & \_         \\ \hline
    8    & B          \\ \hline
    9    & 0          \\ \hline
    \end{tabular}

    \begin{itemize}
        \item 9EF-0EF
        \item 111-211
        \item 700-D00
        \item 0FF-EFF
    \end{itemize}

\end{multicols}


\newpage
\begin{thebibliography}{80}
    \bibitem{http://wikipedia:2019} Wikipedia. http://en.wikipedia.org, last access: 15/04/2019.
    \bibitem{osc}  Silberschatz, Galvin, and Gagne, Operating System Concepts.
    \bibitem{mos} Tanenbaum, Modern Operating Systems.
\end{thebibliography}

\end{document}