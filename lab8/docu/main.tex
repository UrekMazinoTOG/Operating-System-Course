\documentclass[11pt]{article}
\usepackage[english]{babel}
\usepackage[utf8]{vietnam}
\usepackage[margin=2cm]{geometry}
\usepackage[unicode]{hyperref}
\usepackage{color}
\usepackage{array}

\usepackage{fancyhdr}
\pagestyle{fancy}
\fancyhf{}
\rhead{Operating Systems Course}
\lhead{Lab 8}
\rfoot{Page \thepage}
\lfoot{\leftmark}

\begin{document}

\begin{flushleft}
    \begin{tabular}{c}
        \\ \\ \hline \\
        \multicolumn{1}{l}{\textbf{{\Huge Lab 8 Contiguous Memory Allocation}}}		
        \\ \\
        \textbf{{\Huge Course: Operating Systems}}		            
        \\ \\ \hline \\
    \end{tabular}
\end{flushleft}

\selectlanguage{english}

\vspace{0.5cm}

\section*{Exercise 1}
Given six memory partitions of 300 KB, 600 KB, 350 KB, 200 KB, 750 KB, and 125
KB (in order), how would the first-fit, best-fit, and worst-fit algorithms place processes
of size 115 KB, 500 KB, 358 KB, 200 KB, and 375 KB (in order)? Rank the algorithms
in terms of how efficiently they use memory

\begin{table}[!htp]
    \centering
    \def\arraystretch{1.6}

    \begin{tabular}{|l|l|l|}
        \hline
        First fit                    & Best fit                     & Worst fit                    \\ \hline
        115(KB) to 300(KB) partition & 115(KB) to 125(KB) partition & 115(KB) to 750(KB) partition \\ \hline
        500(KB) to 600(KB) partition & 500(KB) to 600(KB) partition & 500(KB) to 600(KB) partition \\ \hline
        358(KB) to 750(KB) partition & 358(KB) to 750(KB) partition & 358(KB) must wait            \\ \hline
        200(KB) to 350(KB) partition & 200(KB) to 200(KB) partition & 200(KB) to 350(KB) partition \\ \hline
        375(KB) must wait            & 375(KB) must wait            & 375(KB) must wait            \\ \hline
    \end{tabular}
\end{table}

\section*{Exercise 2}
Student write a short report that compares the advantages as well as disadvantages
of the allocation algorithms, namely First-Fit, Best-Fit, Worst-Fit.

\begin{table}[!htp]
    \centering
    \def\arraystretch{1.5}

    \begin{tabular}{|m{2cm}|m{6.5cm}|m{7.5cm}|}
    \hline
    Algorithm & Advantage                                                                                                    & Disadvantage                                                                                                                                                  \\ \hline
    First fit & Fastest algorithm because it searches as little as possible.                                                 & The remaining unused memory areas left after allocation become waste if it is too smaller. Thus request for larger memory requirement cannot be accomplished. \\ \hline
    Best fit  & Memory utilization is much better than first fit as it searches the smallest free partition first available. & It is slower and may even tend to fill up memory with tiny useless holes.                                                                                     \\ \hline
    Worst fit & Reduces the rate of production of small gaps.                                                                & If a process requiring larger memory arrives at a later stage then it cannot be accommodated as the largest hole is already split and occupied.               \\ \hline
    \end{tabular}
\end{table}

\newpage
\begin{thebibliography}{80}
    \bibitem{http://wikipedia:2019} Wikipedia. http://en.wikipedia.org, last access: 15/04/2019.
    \bibitem{osc}  Silberschatz, Galvin, and Gagne, Operating System Concepts.
    \bibitem{mos} Tanenbaum, Modern Operating Systems.
\end{thebibliography}

\end{document}