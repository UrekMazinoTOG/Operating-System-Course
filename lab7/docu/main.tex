\documentclass[11pt]{article}
\usepackage[english]{babel}
\usepackage[utf8]{vietnam}
\usepackage[margin=2cm]{geometry}
\usepackage[unicode]{hyperref}
\usepackage{color}
\usepackage{array}

\usepackage{fancyhdr}
\pagestyle{fancy}
\fancyhf{}
\rhead{Operating Systems Course}
\lhead{Lab 7}
\rfoot{Page \thepage}
\lfoot{\leftmark}

\begin{document}

\begin{flushleft}
    \begin{tabular}{c}
        \\ \\ \hline \\
        \multicolumn{1}{l}{\textbf{{\Huge Lab 7 Scheduling}}}		
        \\ \\
        \textbf{{\Huge Course: Operating Systems}}		            
        \\ \\ \hline \\
    \end{tabular}
\end{flushleft}

\selectlanguage{english}

\vspace{1cm}

\subsection*{Exercise 1 : Suppose that the following 
processes arrive for execution at the times indicated. 
Each process will run for the amount of time listed. In 
answering the questions, use non-preemptive scheduling, 
and base all decisions on the information you have at 
the time the decision must be made.}

\begin{table}[!htp]
    \centering            %Center Table%
    \def\arraystretch{2}  %Margin Table%
    
    \begin{tabular}{|c||c|c|}
    \hline
    Process & Arrival time & Burst time \\ \hline
    P1      & 0.0          & 8          \\ \hline
    P2      & 0.4          & 4          \\ \hline
    P3      & 1.0          & 1          \\ \hline
    \end{tabular}
\end{table}

\subsubsection*{a) What is the average turnaround time for these 
processes with the FCFS scheduling algorithm?}

\begin{table}[!htp]
    \centering            %Center Table%
    \def\arraystretch{2}  %Margin Table%
    
    \begin{tabular}{|m{8cm}|m{4cm}|m{1cm}|}
        \hline
        P1 & P2 & P3 \\
        \hline
    \end{tabular}

    \begin{tabular}{m{8cm} m{4cm} m{1cm} l }
        0 & 8 & 12 & 13 \\
    \end{tabular}
    
    \caption{Gantt chart - FCFS}
\end{table}

\begin{itemize}
    \item Turnaround time of process one is $T_{TT1} = 8 - 0 = 8$
    \item Turnaround time of process two is $T_{TT2} = 12 - 0.4 = 11.6$
    \item Turnaround time of process three is $T_{TT3} = 13 - 1 = 12$
\end{itemize}

The average turnaround time is $\overline T = (8 + 11.6 + 12) \div 3 $

\newpage
\subsubsection*{b) What is the average turnaround time 
for these processes with the SJF scheduling algorithm?} 

\begin{table}[!htp]
    \centering            %Center Table%
    \def\arraystretch{2}  %Margin Table%
    
    \begin{tabular}{|m{8cm}|m{1cm}|m{4cm}|}
        \hline
        P1 & P3 & P2 \\
        \hline
    \end{tabular}

    \begin{tabular}{m{8cm} m{1cm} m{4cm} l }
        0 & 8 & 9 & 13 \\
    \end{tabular}
    
    \caption{Gantt chart - SJF - V1}
\end{table}

\begin{itemize}
    \item Turnaround time of process one is $T_{TT1} = 8 - 0 = 8$
    \item Turnaround time of process three is $T_{TT3} = 9 - 1 = 8$
    \item Turnaround time of process two is $T_{TT2} = 13 - 0.4 = 12.6$
\end{itemize}

The average turnaround time is $\overline T = (8 + 8 + 12.6) \div 3 $

\subsubsection*{c) The SJF algorithm is supposed to 
improve performance, but notice that we chose to run process P1 
at time 0 because we did not know that two shorter processes 
would arrive soon. Compute what the average turnaround time will 
be if the CPU is left idle for the first 1 unit and then SJF 
scheduling is used. Remember that processes P1 and P2 are 
waiting during this idle time, so their waiting time may 
increase. This algorithm could be called future-knowledge 
scheduling.} 

\begin{table}[!htp]
    \centering            %Center Table%
    \def\arraystretch{2}  %Margin Table%
    
    \begin{tabular}{|m{1cm}|m{1cm}|m{4cm}|m{8cm}|}
        \hline
        \O & P3 & P2 & P1 \\
        \hline
    \end{tabular}

    \begin{tabular}{m{1cm} m{1cm} m{4cm} m{8cm} l }
        0 & 1 & 2 & 6 & 14 \\
    \end{tabular}
    
    \caption{Gantt chart - SJF - V2}
\end{table}

\begin{itemize}
    \item Turnaround time of process three is $T_{TT3} = 2 - 1 = 1$
    \item Turnaround time of process two is $T_{TT2} = 6 - 0.4 = 5.6$
    \item Turnaround time of process one is $T_{TT1} = 14 - 0 = 14$
\end{itemize}

The average turnaround time is $\overline T = (1 + 5.6 + 14) \div 3 $

\newpage

\subsection*{Exercise 2 : The processes are assumed to have 
arrived in the order P1, P2, P3, P4, P5, all at time 0. Draw 
four Gantt charts that illustrate the execution of these 
processes using the following scheduling algorithms: FCFS, SJF, 
non-preemptive priority (a larger priority number implies a 
higher priority), and RR (quantum = 1). Calculate the average 
waiting time and turnaround time of each scheduling algorithm.}

\begin{table}[!htp]
    \centering            %Center Table%
    \def\arraystretch{2}  %Margin Table%

    \begin{tabular}{|c||c|c|}
    \hline
    Process & Burst time & Priority \\ \hline
    P1      & 8          & 4        \\ \hline
    P2      & 6          & 1        \\ \hline
    P3      & 1          & 2        \\ \hline
    P4      & 9          & 2        \\ \hline
    P5      & 3          & 3        \\ \hline
    \end{tabular}
\end{table}

\subsubsection*{a) The First-Come, First-Served Scheduling (FCFS)}

\begin{table}[!htp]
    \centering            %Center Table%
    \def\arraystretch{2}  %Margin Table%
    
    \begin{tabular}{|m{4cm}|m{3cm}|m{0.5cm}|m{4.5cm}|m{1.5cm}|}
        \hline
        P1 & P2 & P3 & P4 & P5 \\
        \hline
    \end{tabular}

    \begin{tabular}{m{4cm} m{3cm} m{0.5cm} m{4.5cm} m{1.5cm} l}
        0 & 8 & 14 & 15 & 24 & 27 \\
    \end{tabular}
    
    \caption{Gantt chart - FCFS}
\end{table}

\begin{itemize}
    \item The average turnaround time is $\overline T = (8 + 14 + 15 + 24 + 27) \div 5 $
    \item The average waiting time is $\overline T = (0 + 8 + 14 + 15 + 24) \div 5 $
\end{itemize}

\subsubsection*{b) Shortest Job First Scheduling (SJF)} 

\begin{table}[!htp]
    \centering            %Center Table%
    \def\arraystretch{2}  %Margin Table%
    
    \begin{tabular}{|m{0.5cm}|m{1.5cm}|m{3cm}|m{4cm}|m{4.5cm}|}
        \hline
        P3 & P5 & P2 & P4 & P1 \\
        \hline
    \end{tabular}

    \begin{tabular}{m{0.5cm} m{1.5cm} m{3cm} m{4cm} m{4.5cm} l}
        0 & 1 & 4 & 10 & 18 & 27 \\
    \end{tabular}
    
    \caption{Gantt chart - SJF}
\end{table}

\begin{itemize}
    \item The average turnaround time is $\overline T = (1 + 4 + 10 + 18 + 27) \div 5 $
    \item The average waiting time is $\overline T = (0 + 1 + 4 + 10 + 18) \div 5 $
\end{itemize}

\newpage

\subsubsection*{c) Non-preemptive Priority Scheduling}

\begin{table}[!htp]
    \centering            %Center Table%
    \def\arraystretch{2}  %Margin Table%
    
    \begin{tabular}{|m{4cm}|m{1.5cm}|m{4.5cm}|m{0.5cm}|m{3cm}|}
        \hline
        P1 & P5 & P4 & P3 & P2 \\
        \hline
    \end{tabular}

    \begin{tabular}{m{4cm}m{1.5cm}m{4.5cm}m{0.5cm}m{3cm} l}
        0 & 8 & 11 & 20 & 21 & 27 \\
    \end{tabular}
    
    \caption{Gantt chart - non-preemptive priority}
\end{table}

\begin{itemize}
    \item The average turnaround time is $\overline T = (8 + 11 + 20 + 21 + 27) \div 5 $
    \item The average waiting time is $\overline T = (0 + 8 + 11 + 20 + 21) \div 5 $
\end{itemize}

\subsubsection*{d) Round Robin (RR)}

\begin{table}[!htp]
    \centering            %Center Table%
    \def\arraystretch{2}  %Margin Table%
    
    \begin{tabular}{|l|l|l|l|l|l|l|l|l|l|l|l|l|l|l|l|l|l|l|l|l|l|l|l|l|l|l|}
        \hline
        1&2&\textcolor{red}{3}&4&5&
        1&2&4&5&
        1&2&4&\textcolor{red}{5}&
        1&2&4&
        1&2&4&
        1&\textcolor{red}{2}&4&
        1&4&
        \textcolor{red}{1}&4&
        \textcolor{red}{4} \\
        \hline
    \end{tabular}

    \caption{Gantt chart - non-preemptive priority}
\end{table}
\begin{itemize}
    \item The average turnaround time is $\overline T = (3 + 13 + 21 + 25 + 27) \div 5 $
    \item The average waiting time is $\overline T = (17 + 15 + 2 + 18 + 10) \div 5 $
    \begin{itemize}
        \item The waiting time of P1 is $\overline T = 0 + 4 + 3 + 3 + 2 + 2 + 2 + 1 = 17$
        \item The waiting time of P2 is $\overline T = 1 + 4 + 3 + 3 + 2 + 2 = 15$
        \item The waiting time of P3 is $\overline T = 2$
        \item The waiting time of P4 is $\overline T = 3 + 3 + 3 + 3 + 2 + 2 + 1 + 1 = 18$
        \item The waiting time of P5 is $\overline T = 4 + 3 + 3 = 10$
    \end{itemize}
\end{itemize}



\newpage

\begin{thebibliography}{80}
    \bibitem{http://wikipedia:2019} Wikipedia. http://en.wikipedia.org, last access: 10/04/2019.
    \bibitem{osc}  Silberschatz, Galvin, and Gagne, Operating System Concepts.
    \bibitem{mos} Tanenbaum, Modern Operating Systems.
\end{thebibliography}

\end{document}